\documentclass[11pt]{article}
\usepackage{tikz}
\usetikzlibrary{arrows}
\usetikzlibrary{shapes}
\usepackage{pgfmath}
\usepackage{setspace}
\usepackage{amsmath}
\usepackage{array}
\usepackage{hyperref}
\usepackage{enumerate}
\usepackage{enumitem}
\setlist{noitemsep}
\usepackage{listings}
\lstset{language=python}
\usepackage{makeidx}
\usepackage{verbatim}
\usepackage{datetime}

\setlength{\pdfpageheight}{11in}
\setlength{\textheight}{9in}
\setlength{\voffset}{-1in}
\setlength{\oddsidemargin}{0pt}
\setlength{\marginparsep}{0pt}
\setlength{\marginparwidth}{0pt}
\setlength{\marginparpush}{0pt}
\setlength{\textwidth}{6.5in}

\pagestyle{plain}
\makeindex

\title{Weekly Report}
\author{Brad Burkman}
\newdateformat{vardate}{\THEDAY\ \monthname[\THEMONTH]\ \THEYEAR}
\vardate
\date{\today}

\begin{document}
\setlength{\parindent}{20pt}
\begin{spacing}{1.2}
\maketitle
\tableofcontents


\section{Accomplished Tasks}

\begin{itemize}
	\item Discarded articles before 2019, which was about half of them.
	\item Thought about how to narrow focus on articles.  
	\item Reviewed more articles.
\end{itemize}

\section{Questions}

Should I focus on articles for which I can apply Dr. Sun's data set, ignoring articles based on techniques that require data we don't have?  Should I ignore these?

\begin{itemize}
	\item Using ML to analyze video of traffic
	\item Using ML for real-time prediction of driving hazards
	\item Data from sensors in individual vehicles
	\item Data from simulations
\end{itemize}

\section{Future Direction}

\begin{itemize}
	\item The topic of {\it imbalanced learning} seems to be hot.  
	\item ``Future work needs to propose proper methods to supplement these missing data and improve prediction performance.''
\end{itemize}

\section{Articles to Recommend}

``A vehicle speed harmonization strategy for minimizing inter-vehicle crash risks.''

In the last paragraphs, the authors say they are using reinforcement learning for their next step. 

\

``A deep learning based traffic crash severity prediction framework.''

LSU faculty, similar dataset to what we have.  

\

``A systematic assessment of the use of opponent variables, data subsetting and hierarchical specification in two-party crash severity analysis.''

Not ML, but a solid paper.  I'd look to this as a model of how to write well.  The conclusion suggests looking into imbalanced learning.

\


``A Bayesian modeling framework for crash severity effects of active traffic management systems.''

Not ML, but interesting for recommending a comparison between econometric models and ML algorithms.  

\

``A spatiotemporal deep learning approach for citywide short-term crash risk prediction with multi-source data.''

Does a comparison between econometric and ML models.

\

``A hierarchical machine learning classification approach for secondary task identification from observed driving behavior data.''

Interesting in that it looked much more deeply at the data than other studies, looking for correlations between sets of variables.  Also LSU.

\

``A multivariate analysis of environmental effects on road accident occurrence using a balanced bagging approach''

Interesting for handling imbalanced data

\

``A feature learning approach based on XGBoost for driving assessment and risk prediction''

Interesting for its focus on ML methods, not on the dataset.  

\

``A long short-term memory-based framework for crash detection on freeways with traffic data of different temporal resolutions.''

Related to our data and addressing the challenges we'll have with it.  

\

``A review of spatial approaches in road safety.''

I need to read this one as an overview of the field and the jargon.  

\

``A Comprehensive Railroad-Highway Grade Crossing Consolidation Model: A Machine Learning Approach.''

Thorough analysis.  Also LSU.

\

``A systematic review of the association between fault or blame-related attributions and procedures after transport injury and health and work-related outcomes''

Interesting for text mining of crash reports.  

\



%%%%%%%%%%%%%%%%%%
% Index
\clearpage
\addcontentsline{toc}{section}{Index}
\printindex

%%%%%%%%%%%%%%%%
\end{spacing}
\end{document}

%%%%%%%%%%%%
% Useful tools
%%%%%%%%%

\begin{lstlisting}
Put your code here.
\end{lstlisting}

\lstinputlisting[language=python]{source_filename.py}


